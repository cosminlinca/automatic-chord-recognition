\documentclass[a4paper,12pt]{report}
\usepackage{blindtext}
\usepackage[utf8]{inputenc}
\usepackage{amsmath}
\usepackage[margin=1in]{geometry}
\linespread{1.25}
\usepackage[backend=biber]{biblatex}
\usepackage{graphicx}
\usepackage{hyperref}
\usepackage[Symbol]{upgreek}
\usepackage{url}
\usepackage{pythonhighlight}
\usepackage{url}

\begin{document}
\chapter*{Abstract}

The past few years have brought a significant increase in 
interest for acoustic music, 
closely related to the parallel evolution of technology globally, 
with focus on social networks and video streaming.

It was found that the evolution determined a distance between
performers and the musical sheet, as the process of learning through 
online tutorials has become much easier. 
Also, the classical sheet is a musical element with high difficulty, 
being necessary to know some notions of music theory for a full understanding.
Therefore, it proved necessary to use a simple and suggestive notation, 
specific to acoustic music, namely musical tabulature.

Given the existence of these limitations and the desire of guitarists, 
one solution would be the existence of a platform through which acoustic 
parts can be transcribed automatically, directly into a tabular representation,
using guitar chords. 

The first step of the automatic chord recognition system (ACR) is to 
apply a sound processing method, in order to extract important musical 
features, by using a suitable representation in the field, namely chromagram.
This first step is a vital one in the analysis of a musical sample, 
as obtaining a correct representation is closely related to 
the continuous development of the system. 

The second part defines the algorithms that 
underlie the learning processes and differentiate the 
features of some chords from a musical sample, specifically, 
machine learning algorithms. The goal is to gradually arrive at a complex 
and up-to-date machine learning algorithm, able to automatically and independently 
analyze the audio signal and to classify with high 
precision each sequence within an acoustic sample.

In order to highlight the good functioning of the ACR, 
the system will be connected with a mobile application, 
intended for acoustic music enthusiasts. The application will be able to display, 
in real time, the results of the automatic recognition of acoustic musical chords, 
for any desired musical sheet.

\end{document}